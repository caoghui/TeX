%使用中文:方法1
\documentclass[UTF8]{ctexart}
%使用中文:方法2
%\documentclass{article}
%\usepackage{xeCJK}
%\setCJKmainfont{KaiTi}
%
%这里是导言区
%
\title{C++拷贝控制}
\author{Bruce}
\date{\today}
\usepackage{graphicx}
%设置页边距
\usepackage{geometry}
%\geometry{papersize={20cm,15cm}}
\geometry{left=1cm,right=2cm,top=3cm,bottom=4cm}
%设置页眉页脚
\usepackage{fancyhdr}
\pagestyle{fancy}
\lhead{\author}
\chead{\today}
\rhead{}
\lfoot{}
\cfoot{\thepage}
\rfoot{}
\renewcommand{\headrulewidth}{0.4pt}
\renewcommand{\headwidth}{\textwidth}
\renewcommand{\footrulewidth}{0pt}

\usepackage[american]{babel}
\usepackage{microtype}

%行间距
\usepackage{setspace}
\onehalfspacing

%段间距
\addtolength{\parskip}{.4em}

%插入源代码 method 1
\usepackage{minted}

%插入源代码 method 2
\usepackage{listings}
\usepackage{xcolor}
\definecolor{dkgreen}{rgb}{0,0.6,0}
\definecolor{gray}{rgb}{0.5,0.5,0.5}
\definecolor{mauve}{rgb}{0.58,0,0.82}
\lstset{ 
    language=c++,
    showspaces=false,
    showtabs=false,
    tabsize=4,
    frame=shadowbox, %single
    framerule=1pt,
    framexleftmargin=5mm,
    framexrightmargin=5mm,
    framextopmargin=5mm,
    framexbottommargin=5mm,
    numbers=left,
    numberstyle=\small\color{gray}, %\small    
    captionpos=b,
    rulecolor=\color{black},
	 rulesepcolor= \color{ red!20!green!20!blue!20} ,
	 backgroundcolor=\color{white},
    escapeinside=``, % 英文分号中可写入中文
	 basicstyle=\footnotesize, %\tt
	 directivestyle=\tt,
	 identifierstyle=\tt,
	 commentstyle=\tt,
	 stringstyle=\tt,
	 keywordstyle=\color{blue}\tt
}

%设置代码字体
\setmainfont{Times New Roman}
\setsansfont{Arial}
\setmonofont{Courier New}

%算法
\usepackage[ruled]{algorithm2e}

\begin{document}
\maketitle
\tableofcontents

\section{你好中国}
中国在East Asia.
\subsection{Hello Beijing}
北京是capital of China.
\subsubsection{Hello Dongcheng District}
\paragraph{Tian'anmen Square}
is in the center of Beijing
\subparagraph{Chairman Mao}
is in the center of 天安门广场.
\subsection{Hello 山东}
\paragraph{山东大学} is one of the best university in 山东。\newline
Hello World! 你好啊!!!
这行是从vscode中添加的


\begin{minted}[linenos,frame=lines]{c++}
int main()
{
    cout << "Hello LaTeX" << endl;
    return 0;
}
\end{minted}

\begin{lstlisting}[caption= hello world]
#include <iostream>
int main(int argc, char** argv)
{
    return 0;
}
\end{lstlisting}


\begin{verbatim}
#include <iostream>
int main(int argc, char** argv)
{
    std::cout<<"Hello World"<<endl;
    return 0;
}
\end{verbatim}

\begin{algorithm}[H]
\caption{How to write algorithms}
\KwIn{this text}
\KwOut{How to write algorithm with \LaTeX2e}
initialization\;
\While{not at end of this document}{
    read current\;
    \eIf{understand}{
        go to next section\;
        current section becomes this one\;
    }{
        go back to the beginning of current section\;
    }
}

\end{algorithm}

\begin{table}[htbp]
\centering
\begin{tabular}{|l|c|r|}
\hline
操作系统 & 发行版 & 编辑器\\
\hline
Windows & MikTeX & TexMakerX \\
\hline
Unix/Linux & teTeX & Kile \\
\hline
MacOS  & MaxTeX  & TeXShop \\
\hline
通用 & TeX Live & TeXworks \\
\hline
\end{tabular}
\caption{Editor}
\label{tbl:editor}
\end{table}



\end{document}